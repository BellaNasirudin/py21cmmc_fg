
% Default to the notebook output style

    


% Inherit from the specified cell style.




    
\documentclass[11pt]{article}

    
    
    \usepackage[T1]{fontenc}
    % Nicer default font (+ math font) than Computer Modern for most use cases
    \usepackage{mathpazo}

    % Basic figure setup, for now with no caption control since it's done
    % automatically by Pandoc (which extracts ![](path) syntax from Markdown).
    \usepackage{graphicx}
    % We will generate all images so they have a width \maxwidth. This means
    % that they will get their normal width if they fit onto the page, but
    % are scaled down if they would overflow the margins.
    \makeatletter
    \def\maxwidth{\ifdim\Gin@nat@width>\linewidth\linewidth
    \else\Gin@nat@width\fi}
    \makeatother
    \let\Oldincludegraphics\includegraphics
    % Set max figure width to be 80% of text width, for now hardcoded.
    \renewcommand{\includegraphics}[1]{\Oldincludegraphics[width=.8\maxwidth]{#1}}
    % Ensure that by default, figures have no caption (until we provide a
    % proper Figure object with a Caption API and a way to capture that
    % in the conversion process - todo).
    \usepackage{caption}
    \DeclareCaptionLabelFormat{nolabel}{}
    \captionsetup{labelformat=nolabel}

    \usepackage{adjustbox} % Used to constrain images to a maximum size 
    \usepackage{xcolor} % Allow colors to be defined
    \usepackage{enumerate} % Needed for markdown enumerations to work
    \usepackage{geometry} % Used to adjust the document margins
    \usepackage{amsmath} % Equations
    \usepackage{amssymb} % Equations
    \usepackage{textcomp} % defines textquotesingle
    % Hack from http://tex.stackexchange.com/a/47451/13684:
    \AtBeginDocument{%
        \def\PYZsq{\textquotesingle}% Upright quotes in Pygmentized code
    }
    \usepackage{upquote} % Upright quotes for verbatim code
    \usepackage{eurosym} % defines \euro
    \usepackage[mathletters]{ucs} % Extended unicode (utf-8) support
    \usepackage[utf8x]{inputenc} % Allow utf-8 characters in the tex document
    \usepackage{fancyvrb} % verbatim replacement that allows latex
    \usepackage{grffile} % extends the file name processing of package graphics 
                         % to support a larger range 
    % The hyperref package gives us a pdf with properly built
    % internal navigation ('pdf bookmarks' for the table of contents,
    % internal cross-reference links, web links for URLs, etc.)
    \usepackage{hyperref}
    \usepackage{longtable} % longtable support required by pandoc >1.10
    \usepackage{booktabs}  % table support for pandoc > 1.12.2
    \usepackage[inline]{enumitem} % IRkernel/repr support (it uses the enumerate* environment)
    \usepackage[normalem]{ulem} % ulem is needed to support strikethroughs (\sout)
                                % normalem makes italics be italics, not underlines
    

    
    
    % Colors for the hyperref package
    \definecolor{urlcolor}{rgb}{0,.145,.698}
    \definecolor{linkcolor}{rgb}{.71,0.21,0.01}
    \definecolor{citecolor}{rgb}{.12,.54,.11}

    % ANSI colors
    \definecolor{ansi-black}{HTML}{3E424D}
    \definecolor{ansi-black-intense}{HTML}{282C36}
    \definecolor{ansi-red}{HTML}{E75C58}
    \definecolor{ansi-red-intense}{HTML}{B22B31}
    \definecolor{ansi-green}{HTML}{00A250}
    \definecolor{ansi-green-intense}{HTML}{007427}
    \definecolor{ansi-yellow}{HTML}{DDB62B}
    \definecolor{ansi-yellow-intense}{HTML}{B27D12}
    \definecolor{ansi-blue}{HTML}{208FFB}
    \definecolor{ansi-blue-intense}{HTML}{0065CA}
    \definecolor{ansi-magenta}{HTML}{D160C4}
    \definecolor{ansi-magenta-intense}{HTML}{A03196}
    \definecolor{ansi-cyan}{HTML}{60C6C8}
    \definecolor{ansi-cyan-intense}{HTML}{258F8F}
    \definecolor{ansi-white}{HTML}{C5C1B4}
    \definecolor{ansi-white-intense}{HTML}{A1A6B2}

    % commands and environments needed by pandoc snippets
    % extracted from the output of `pandoc -s`
    \providecommand{\tightlist}{%
      \setlength{\itemsep}{0pt}\setlength{\parskip}{0pt}}
    \DefineVerbatimEnvironment{Highlighting}{Verbatim}{commandchars=\\\{\}}
    % Add ',fontsize=\small' for more characters per line
    \newenvironment{Shaded}{}{}
    \newcommand{\KeywordTok}[1]{\textcolor[rgb]{0.00,0.44,0.13}{\textbf{{#1}}}}
    \newcommand{\DataTypeTok}[1]{\textcolor[rgb]{0.56,0.13,0.00}{{#1}}}
    \newcommand{\DecValTok}[1]{\textcolor[rgb]{0.25,0.63,0.44}{{#1}}}
    \newcommand{\BaseNTok}[1]{\textcolor[rgb]{0.25,0.63,0.44}{{#1}}}
    \newcommand{\FloatTok}[1]{\textcolor[rgb]{0.25,0.63,0.44}{{#1}}}
    \newcommand{\CharTok}[1]{\textcolor[rgb]{0.25,0.44,0.63}{{#1}}}
    \newcommand{\StringTok}[1]{\textcolor[rgb]{0.25,0.44,0.63}{{#1}}}
    \newcommand{\CommentTok}[1]{\textcolor[rgb]{0.38,0.63,0.69}{\textit{{#1}}}}
    \newcommand{\OtherTok}[1]{\textcolor[rgb]{0.00,0.44,0.13}{{#1}}}
    \newcommand{\AlertTok}[1]{\textcolor[rgb]{1.00,0.00,0.00}{\textbf{{#1}}}}
    \newcommand{\FunctionTok}[1]{\textcolor[rgb]{0.02,0.16,0.49}{{#1}}}
    \newcommand{\RegionMarkerTok}[1]{{#1}}
    \newcommand{\ErrorTok}[1]{\textcolor[rgb]{1.00,0.00,0.00}{\textbf{{#1}}}}
    \newcommand{\NormalTok}[1]{{#1}}
    
    % Additional commands for more recent versions of Pandoc
    \newcommand{\ConstantTok}[1]{\textcolor[rgb]{0.53,0.00,0.00}{{#1}}}
    \newcommand{\SpecialCharTok}[1]{\textcolor[rgb]{0.25,0.44,0.63}{{#1}}}
    \newcommand{\VerbatimStringTok}[1]{\textcolor[rgb]{0.25,0.44,0.63}{{#1}}}
    \newcommand{\SpecialStringTok}[1]{\textcolor[rgb]{0.73,0.40,0.53}{{#1}}}
    \newcommand{\ImportTok}[1]{{#1}}
    \newcommand{\DocumentationTok}[1]{\textcolor[rgb]{0.73,0.13,0.13}{\textit{{#1}}}}
    \newcommand{\AnnotationTok}[1]{\textcolor[rgb]{0.38,0.63,0.69}{\textbf{\textit{{#1}}}}}
    \newcommand{\CommentVarTok}[1]{\textcolor[rgb]{0.38,0.63,0.69}{\textbf{\textit{{#1}}}}}
    \newcommand{\VariableTok}[1]{\textcolor[rgb]{0.10,0.09,0.49}{{#1}}}
    \newcommand{\ControlFlowTok}[1]{\textcolor[rgb]{0.00,0.44,0.13}{\textbf{{#1}}}}
    \newcommand{\OperatorTok}[1]{\textcolor[rgb]{0.40,0.40,0.40}{{#1}}}
    \newcommand{\BuiltInTok}[1]{{#1}}
    \newcommand{\ExtensionTok}[1]{{#1}}
    \newcommand{\PreprocessorTok}[1]{\textcolor[rgb]{0.74,0.48,0.00}{{#1}}}
    \newcommand{\AttributeTok}[1]{\textcolor[rgb]{0.49,0.56,0.16}{{#1}}}
    \newcommand{\InformationTok}[1]{\textcolor[rgb]{0.38,0.63,0.69}{\textbf{\textit{{#1}}}}}
    \newcommand{\WarningTok}[1]{\textcolor[rgb]{0.38,0.63,0.69}{\textbf{\textit{{#1}}}}}
    
    
    % Define a nice break command that doesn't care if a line doesn't already
    % exist.
    \def\br{\hspace*{\fill} \\* }
    % Math Jax compatability definitions
    \def\gt{>}
    \def\lt{<}
    % Document parameters
    \title{noise\_power\_derivation}
    
    
    

    % Pygments definitions
    
\makeatletter
\def\PY@reset{\let\PY@it=\relax \let\PY@bf=\relax%
    \let\PY@ul=\relax \let\PY@tc=\relax%
    \let\PY@bc=\relax \let\PY@ff=\relax}
\def\PY@tok#1{\csname PY@tok@#1\endcsname}
\def\PY@toks#1+{\ifx\relax#1\empty\else%
    \PY@tok{#1}\expandafter\PY@toks\fi}
\def\PY@do#1{\PY@bc{\PY@tc{\PY@ul{%
    \PY@it{\PY@bf{\PY@ff{#1}}}}}}}
\def\PY#1#2{\PY@reset\PY@toks#1+\relax+\PY@do{#2}}

\expandafter\def\csname PY@tok@w\endcsname{\def\PY@tc##1{\textcolor[rgb]{0.73,0.73,0.73}{##1}}}
\expandafter\def\csname PY@tok@c\endcsname{\let\PY@it=\textit\def\PY@tc##1{\textcolor[rgb]{0.25,0.50,0.50}{##1}}}
\expandafter\def\csname PY@tok@cp\endcsname{\def\PY@tc##1{\textcolor[rgb]{0.74,0.48,0.00}{##1}}}
\expandafter\def\csname PY@tok@k\endcsname{\let\PY@bf=\textbf\def\PY@tc##1{\textcolor[rgb]{0.00,0.50,0.00}{##1}}}
\expandafter\def\csname PY@tok@kp\endcsname{\def\PY@tc##1{\textcolor[rgb]{0.00,0.50,0.00}{##1}}}
\expandafter\def\csname PY@tok@kt\endcsname{\def\PY@tc##1{\textcolor[rgb]{0.69,0.00,0.25}{##1}}}
\expandafter\def\csname PY@tok@o\endcsname{\def\PY@tc##1{\textcolor[rgb]{0.40,0.40,0.40}{##1}}}
\expandafter\def\csname PY@tok@ow\endcsname{\let\PY@bf=\textbf\def\PY@tc##1{\textcolor[rgb]{0.67,0.13,1.00}{##1}}}
\expandafter\def\csname PY@tok@nb\endcsname{\def\PY@tc##1{\textcolor[rgb]{0.00,0.50,0.00}{##1}}}
\expandafter\def\csname PY@tok@nf\endcsname{\def\PY@tc##1{\textcolor[rgb]{0.00,0.00,1.00}{##1}}}
\expandafter\def\csname PY@tok@nc\endcsname{\let\PY@bf=\textbf\def\PY@tc##1{\textcolor[rgb]{0.00,0.00,1.00}{##1}}}
\expandafter\def\csname PY@tok@nn\endcsname{\let\PY@bf=\textbf\def\PY@tc##1{\textcolor[rgb]{0.00,0.00,1.00}{##1}}}
\expandafter\def\csname PY@tok@ne\endcsname{\let\PY@bf=\textbf\def\PY@tc##1{\textcolor[rgb]{0.82,0.25,0.23}{##1}}}
\expandafter\def\csname PY@tok@nv\endcsname{\def\PY@tc##1{\textcolor[rgb]{0.10,0.09,0.49}{##1}}}
\expandafter\def\csname PY@tok@no\endcsname{\def\PY@tc##1{\textcolor[rgb]{0.53,0.00,0.00}{##1}}}
\expandafter\def\csname PY@tok@nl\endcsname{\def\PY@tc##1{\textcolor[rgb]{0.63,0.63,0.00}{##1}}}
\expandafter\def\csname PY@tok@ni\endcsname{\let\PY@bf=\textbf\def\PY@tc##1{\textcolor[rgb]{0.60,0.60,0.60}{##1}}}
\expandafter\def\csname PY@tok@na\endcsname{\def\PY@tc##1{\textcolor[rgb]{0.49,0.56,0.16}{##1}}}
\expandafter\def\csname PY@tok@nt\endcsname{\let\PY@bf=\textbf\def\PY@tc##1{\textcolor[rgb]{0.00,0.50,0.00}{##1}}}
\expandafter\def\csname PY@tok@nd\endcsname{\def\PY@tc##1{\textcolor[rgb]{0.67,0.13,1.00}{##1}}}
\expandafter\def\csname PY@tok@s\endcsname{\def\PY@tc##1{\textcolor[rgb]{0.73,0.13,0.13}{##1}}}
\expandafter\def\csname PY@tok@sd\endcsname{\let\PY@it=\textit\def\PY@tc##1{\textcolor[rgb]{0.73,0.13,0.13}{##1}}}
\expandafter\def\csname PY@tok@si\endcsname{\let\PY@bf=\textbf\def\PY@tc##1{\textcolor[rgb]{0.73,0.40,0.53}{##1}}}
\expandafter\def\csname PY@tok@se\endcsname{\let\PY@bf=\textbf\def\PY@tc##1{\textcolor[rgb]{0.73,0.40,0.13}{##1}}}
\expandafter\def\csname PY@tok@sr\endcsname{\def\PY@tc##1{\textcolor[rgb]{0.73,0.40,0.53}{##1}}}
\expandafter\def\csname PY@tok@ss\endcsname{\def\PY@tc##1{\textcolor[rgb]{0.10,0.09,0.49}{##1}}}
\expandafter\def\csname PY@tok@sx\endcsname{\def\PY@tc##1{\textcolor[rgb]{0.00,0.50,0.00}{##1}}}
\expandafter\def\csname PY@tok@m\endcsname{\def\PY@tc##1{\textcolor[rgb]{0.40,0.40,0.40}{##1}}}
\expandafter\def\csname PY@tok@gh\endcsname{\let\PY@bf=\textbf\def\PY@tc##1{\textcolor[rgb]{0.00,0.00,0.50}{##1}}}
\expandafter\def\csname PY@tok@gu\endcsname{\let\PY@bf=\textbf\def\PY@tc##1{\textcolor[rgb]{0.50,0.00,0.50}{##1}}}
\expandafter\def\csname PY@tok@gd\endcsname{\def\PY@tc##1{\textcolor[rgb]{0.63,0.00,0.00}{##1}}}
\expandafter\def\csname PY@tok@gi\endcsname{\def\PY@tc##1{\textcolor[rgb]{0.00,0.63,0.00}{##1}}}
\expandafter\def\csname PY@tok@gr\endcsname{\def\PY@tc##1{\textcolor[rgb]{1.00,0.00,0.00}{##1}}}
\expandafter\def\csname PY@tok@ge\endcsname{\let\PY@it=\textit}
\expandafter\def\csname PY@tok@gs\endcsname{\let\PY@bf=\textbf}
\expandafter\def\csname PY@tok@gp\endcsname{\let\PY@bf=\textbf\def\PY@tc##1{\textcolor[rgb]{0.00,0.00,0.50}{##1}}}
\expandafter\def\csname PY@tok@go\endcsname{\def\PY@tc##1{\textcolor[rgb]{0.53,0.53,0.53}{##1}}}
\expandafter\def\csname PY@tok@gt\endcsname{\def\PY@tc##1{\textcolor[rgb]{0.00,0.27,0.87}{##1}}}
\expandafter\def\csname PY@tok@err\endcsname{\def\PY@bc##1{\setlength{\fboxsep}{0pt}\fcolorbox[rgb]{1.00,0.00,0.00}{1,1,1}{\strut ##1}}}
\expandafter\def\csname PY@tok@kc\endcsname{\let\PY@bf=\textbf\def\PY@tc##1{\textcolor[rgb]{0.00,0.50,0.00}{##1}}}
\expandafter\def\csname PY@tok@kd\endcsname{\let\PY@bf=\textbf\def\PY@tc##1{\textcolor[rgb]{0.00,0.50,0.00}{##1}}}
\expandafter\def\csname PY@tok@kn\endcsname{\let\PY@bf=\textbf\def\PY@tc##1{\textcolor[rgb]{0.00,0.50,0.00}{##1}}}
\expandafter\def\csname PY@tok@kr\endcsname{\let\PY@bf=\textbf\def\PY@tc##1{\textcolor[rgb]{0.00,0.50,0.00}{##1}}}
\expandafter\def\csname PY@tok@bp\endcsname{\def\PY@tc##1{\textcolor[rgb]{0.00,0.50,0.00}{##1}}}
\expandafter\def\csname PY@tok@fm\endcsname{\def\PY@tc##1{\textcolor[rgb]{0.00,0.00,1.00}{##1}}}
\expandafter\def\csname PY@tok@vc\endcsname{\def\PY@tc##1{\textcolor[rgb]{0.10,0.09,0.49}{##1}}}
\expandafter\def\csname PY@tok@vg\endcsname{\def\PY@tc##1{\textcolor[rgb]{0.10,0.09,0.49}{##1}}}
\expandafter\def\csname PY@tok@vi\endcsname{\def\PY@tc##1{\textcolor[rgb]{0.10,0.09,0.49}{##1}}}
\expandafter\def\csname PY@tok@vm\endcsname{\def\PY@tc##1{\textcolor[rgb]{0.10,0.09,0.49}{##1}}}
\expandafter\def\csname PY@tok@sa\endcsname{\def\PY@tc##1{\textcolor[rgb]{0.73,0.13,0.13}{##1}}}
\expandafter\def\csname PY@tok@sb\endcsname{\def\PY@tc##1{\textcolor[rgb]{0.73,0.13,0.13}{##1}}}
\expandafter\def\csname PY@tok@sc\endcsname{\def\PY@tc##1{\textcolor[rgb]{0.73,0.13,0.13}{##1}}}
\expandafter\def\csname PY@tok@dl\endcsname{\def\PY@tc##1{\textcolor[rgb]{0.73,0.13,0.13}{##1}}}
\expandafter\def\csname PY@tok@s2\endcsname{\def\PY@tc##1{\textcolor[rgb]{0.73,0.13,0.13}{##1}}}
\expandafter\def\csname PY@tok@sh\endcsname{\def\PY@tc##1{\textcolor[rgb]{0.73,0.13,0.13}{##1}}}
\expandafter\def\csname PY@tok@s1\endcsname{\def\PY@tc##1{\textcolor[rgb]{0.73,0.13,0.13}{##1}}}
\expandafter\def\csname PY@tok@mb\endcsname{\def\PY@tc##1{\textcolor[rgb]{0.40,0.40,0.40}{##1}}}
\expandafter\def\csname PY@tok@mf\endcsname{\def\PY@tc##1{\textcolor[rgb]{0.40,0.40,0.40}{##1}}}
\expandafter\def\csname PY@tok@mh\endcsname{\def\PY@tc##1{\textcolor[rgb]{0.40,0.40,0.40}{##1}}}
\expandafter\def\csname PY@tok@mi\endcsname{\def\PY@tc##1{\textcolor[rgb]{0.40,0.40,0.40}{##1}}}
\expandafter\def\csname PY@tok@il\endcsname{\def\PY@tc##1{\textcolor[rgb]{0.40,0.40,0.40}{##1}}}
\expandafter\def\csname PY@tok@mo\endcsname{\def\PY@tc##1{\textcolor[rgb]{0.40,0.40,0.40}{##1}}}
\expandafter\def\csname PY@tok@ch\endcsname{\let\PY@it=\textit\def\PY@tc##1{\textcolor[rgb]{0.25,0.50,0.50}{##1}}}
\expandafter\def\csname PY@tok@cm\endcsname{\let\PY@it=\textit\def\PY@tc##1{\textcolor[rgb]{0.25,0.50,0.50}{##1}}}
\expandafter\def\csname PY@tok@cpf\endcsname{\let\PY@it=\textit\def\PY@tc##1{\textcolor[rgb]{0.25,0.50,0.50}{##1}}}
\expandafter\def\csname PY@tok@c1\endcsname{\let\PY@it=\textit\def\PY@tc##1{\textcolor[rgb]{0.25,0.50,0.50}{##1}}}
\expandafter\def\csname PY@tok@cs\endcsname{\let\PY@it=\textit\def\PY@tc##1{\textcolor[rgb]{0.25,0.50,0.50}{##1}}}

\def\PYZbs{\char`\\}
\def\PYZus{\char`\_}
\def\PYZob{\char`\{}
\def\PYZcb{\char`\}}
\def\PYZca{\char`\^}
\def\PYZam{\char`\&}
\def\PYZlt{\char`\<}
\def\PYZgt{\char`\>}
\def\PYZsh{\char`\#}
\def\PYZpc{\char`\%}
\def\PYZdl{\char`\$}
\def\PYZhy{\char`\-}
\def\PYZsq{\char`\'}
\def\PYZdq{\char`\"}
\def\PYZti{\char`\~}
% for compatibility with earlier versions
\def\PYZat{@}
\def\PYZlb{[}
\def\PYZrb{]}
\makeatother


    % Exact colors from NB
    \definecolor{incolor}{rgb}{0.0, 0.0, 0.5}
    \definecolor{outcolor}{rgb}{0.545, 0.0, 0.0}



    
    % Prevent overflowing lines due to hard-to-break entities
    \sloppy 
    % Setup hyperref package
    \hypersetup{
      breaklinks=true,  % so long urls are correctly broken across lines
      colorlinks=true,
      urlcolor=urlcolor,
      linkcolor=linkcolor,
      citecolor=citecolor,
      }
    % Slightly bigger margins than the latex defaults
    
    \geometry{verbose,tmargin=1in,bmargin=1in,lmargin=1in,rmargin=1in}
    
    

    \begin{document}
    
    
    \maketitle
    
    

    
    \hypertarget{derivation-of-expectation-and-variance-of-power-from-thermal-noise}{%
\section{Derivation of Expectation and Variance of Power from Thermal
Noise}\label{derivation-of-expectation-and-variance-of-power-from-thermal-noise}}

    \hypertarget{preliminaries}{%
\subsection{Preliminaries}\label{preliminaries}}

    For the general likelihood of the 2D power (or even 1D power), one needs
to know the contribution to the power (and its uncertainty) from thermal
noise. In fact, of course, the thermal noise is added in a non-Gaussian
manner (same as everything else, really) because it is strictly
positive. However, we assume that it is at least \emph{close} to
Gaussian, so that we can describe it purely by its mean and variance.

    The other thing that is often assumed is that the thermal noise can be
treated independently of all other signals in the chain (eg. EoR and
FG). We will show that while this is a reasonable assumption when the
expected power (from FG and EoR) is much less than the variance of the
noise, it does not hold in general. Ignoring this fact leads to
consistent under-estimation of the total variance of the power spectrum.
We will first perform the calculation assuming that the FG, EoR and
noise can be separated, before doing the full derivation.

    Throughout the following, we assume that the variance of the
\emph{magnitude} of a visibility from thermal noise is \(\sigma^2\).
This requires that the variance of the real and imaginary components,
\(\sigma^2_\mathcal{R}, \sigma^2_\mathcal{I} \equiv \sigma^2 /2\).

    \hypertarget{independent-calculation}{%
\subsection{Independent Calculation}\label{independent-calculation}}

    Since the thermal noise is independent of everything else, we can
\emph{just} deal with it, and neglect everything else, then add it in at
the end. So let's begin.

Let every noise visibility, at frequency \(\nu\), be called
\(V_{i,\nu}\) (we don't label it for being noise, as we won't deal with
any other type of visibility in this document). Then the visibility of a
\(uv\nu\) grid cell is

\begin{equation}
    V_{uv\nu} = \frac{1}{n_{uv\nu}}\sum_i V_{i,\nu} \delta_{u_i,v_i \sim uv},
\end{equation}

where the \(\delta\) function really just means ``only count baselines
in the \(uv\) cell'', and \(n_{uv\nu}\) is just the number of baselines
in the \(uv\) cell at that frequency. Then of course the fourier-space
visibility is

\begin{equation}
    V_{uv\eta} = \Delta\nu \sum e^{-2\pi i \eta \nu} V_{uv\nu} \phi_\nu,
\end{equation}

where \(\phi_\nu\) is some applied taper and \(\Delta\nu\) is the
frequency channel width.

Thus we can get the expectation and variance of \(V_{uv\eta}\):

\begin{align}
    \langle V_{uv\eta} \rangle &= 0, \\
    {\rm Var}(V_{uv\eta}) &= \sigma^2 \sum_\nu \frac{(\Delta\nu)^2 \phi_\nu^2}{n_{uv\nu}} \equiv \sigma^2 / \tilde{n}_{uv}.
\end{align}

With
\(\tilde{n}_{uv} = \left[\sum_\nu \frac{(\Delta\nu)^2 \phi_\nu^2}{n_{uv\nu}}\right]^{-1}\)
an ``effective'' number of baselines in a \(uv\eta\) cell.

    Now then, we have \(P_{uv\eta} = |V_{uv\eta}|^2\), and we need to
evaluate the mean and variance of the power. For this, we simply use the
relation that the mean and variance of a Gaussian variable squared are
\(\sigma^2\) and \(2\sigma^4\) respectively, so:

\begin{align}
    \langle P_{uv\eta} \rangle &= \sigma^2 / \tilde{n}_{uv} \equiv {\rm Var}(V_{uv\eta}), \\
    {\rm Var}(P_{uv\eta}) &= \frac{2 \sigma^4_\mathcal{R} + 2\sigma^4_{\mathcal{I}}}{\tilde{n}^2_{uv}} \\
    &= \sigma^4 / \tilde{n}^2_{uv} \equiv \langle P_{uv\eta} \rangle ^2.
\end{align}

    Now, the power spectrum is circularly averaged. When doing this, we are
careful to use the weights of each grid point, which is properly the
inverse of the standard deviation of each bin:

\begin{equation}
    P_{u\eta} = \frac{\sum P_{uv\eta} w_{uv\eta}}{\sum w_{uv\eta}},
\end{equation}

where the sum is over all cells within an annulus, and

\begin{equation}
    w_{uv\eta}  = \frac{\tilde{n}_{uv}}{\sigma^2}.
\end{equation}

    Thus we have

\begin{equation}
    P_{u\eta} = \frac{\sum \tilde{n}_{uv} P_{uv\eta} }{\sum  \tilde{n}_{uv}}.
\end{equation}

    Finally, we need to calculate the mean and variance of this quantity:

\begin{align}
    \langle P_{u\eta} \rangle &= \frac{\sum \sigma^2}{\sum \tilde{n}_{uv}} \\
    &=\sigma^2 \frac{n_u}{\sum \tilde{n}_{uv}},
\end{align}

where \(n_u\) is the number of \(uv\) cells in the \(u\) annulus. This
averages down roughly as \(1/\bar{n}_{uv}\). Then the variance:

\begin{align}
    {\rm Var}(P_{u\eta}) &= \frac{\sigma^4 n_u}{(\sum \tilde{n}_{uv})^2}.
\end{align}

    \hypertarget{general-derivation}{%
\subsection{General Derivation}\label{general-derivation}}

    Here we do a (more) general derivation, where we acknowledge that there
is EoR signal (with non-zero expectation) that may contribute to the
variance of the power. For simplicity, we attempt to reproduce the
conditions under which the \texttt{test\_thermal\_noise} script is being
run: i.e.~zero foregrounds, and deterministic signal. \textbf{NOTE: in
general neither of these assumptions is true, and this derivwation will
have to be done again more generally}.

    Let \(V_{i,\nu} = V_{S,i,\nu} + V_{N,i,\nu}\), where \(S\) stands for
\emph{signal}.

    We note that every operation up to the squaring of the visibilities is
linear, and therefore the equations are valid for the noise term (and
other terms can be derived similarly). Thus we require only to do
calculations from \$\langle P\_\{uv\eta\} \rangle onwards. We have, in
particular,

\begin{align}
    P_{uv\eta} &= |V_{uv\eta}|^2 \\
    &= |V_{S,uv\eta} + V_{N, uv\eta}|^2 \\
    &= ||V_{S, uv\eta}|^2 + |V_{N, uv\eta}|^2 + 2\mathcal{Re}\left\{V_{S, uv\eta}V^*_{N, uv\eta}\right\} 
\end{align}

    For the expectation of the power, each term is of course treated
separately, and all terms involving a single \(V_N\) disappear, as
\(\langle X Y \rangle\) for \(\langle X \rangle = 0\) and \(X,Y\)
independent is zero. Thus the expectation of the power is

\begin{align}
    \langle P_{uv\eta} \rangle &= {\rm Var}(V_{S, uv\eta}) + \frac{\sigma^2}{\tilde{n}_{uv}} \\
    &= \langle P_{N,uv\eta} \rangle,
\end{align}

since the \emph{variance} of the signal is zero (as it is considered
deterministic here).

    However, for the variance, we have the independent parts acting in the
standard manner, but recall that for independent \(X,Y\),

\begin{equation}
    {\rm Var}(XY) = {\rm Var}(X){\rm Var}(Y) + {\rm Var}(X) |\langle Y \rangle|^2 + {\rm Var}(Y) |\langle X \rangle|^2.
\end{equation}

    So we have \begin{align}
    {\rm Var}(P_{uv\eta}) = &\langle P_{N, uv\eta} \rangle^2 + \langle P_{S, uv\eta} \rangle^2 + 2 {\rm Var}(V_{N,uv\eta})P_{S,uv\eta} + \\
        & 2 {\rm Cov}(|V_{S,uv\eta}|^2, \mathcal{Re}\left\{V_{N, uv\eta}V^*_{S, uv\eta}\right\}) + 2 {\rm Cov}(|V_{N,uv\eta}|^2, \mathcal{Re}\left\{V_{N, uv\eta}V^*_{S, uv\eta}\right\}) 
\end{align}

    Notice that the last term on the first line needs reasonably careful
handling of the real/imaginary components, and recalling that \(S\) is
deterministic.

    Recall also that

\begin{align}
{\rm Cov}[XX, XY] &= E[X^3]E[Y] - E[X^2]E[X]E[Y]  \\
\end{align}

for independent \(X,Y\). But this is not exactly what we have. Instead,
we have\\
\begin{align}
{\rm Cov}[|X|^2, \mathcal{Re} (XY)] &= E[X_\mathcal{R}^3]E[Y_\mathcal{R}] - E[X_\mathcal{R}^2]E[X_\mathcal{R}]E[Y_\mathcal{R}] -  E[X_\mathcal{I}^3]E[Y_\mathcal{I}] + E[X_\mathcal{I}^2]E[X_\mathcal{I}]E[Y_\mathcal{I}].
\end{align}

Regardless, this removes all covariance terms, as its noise part appears
to the odd power in each, so we are left with

\begin{align}
    {\rm Var}(P_{uv\eta}) &= \langle P_{N, uv\eta} \rangle^2 + \langle P_{S, uv\eta} \rangle^2 + 2{\rm Var}(V_{N,uv\eta})|\langle V_{S,uv\eta} \rangle|^2 \\
    &= \langle P_{N, uv\eta} \rangle \left[ \langle P_{N, uv\eta} \rangle + 2|\langle V_{S,uv\eta} \rangle|^2 \right] + \langle P_{S, uv\eta} \rangle^2
    \end{align}

    Since the signal is deterministic, we can do away with its expectation
signs, and note that \(|V_S|^2 \equiv P_S\):

\begin{align}
    {\rm Var}(P_{uv\eta}) &= \langle P_{N, uv\eta} \rangle \left[ \langle P_{N, uv\eta} \rangle + 2P_{S,uv\eta} \right] + P_{S, uv\eta}^2
    \end{align}

    Thus, when \(\langle P_N \rangle \gg 2 P_S\), the solution is the same
as the above independent case. But this will not be true where the
signal power is large compared to the noise (for which we must assume
some bins exist).

    \textbf{The question is: is this correct? At the moment, I can only see
how this would add \emph{more} power in some bins, but our numerical
estimates suggest that the analytical variance is already too high}


    % Add a bibliography block to the postdoc
    
    
    
    \end{document}
